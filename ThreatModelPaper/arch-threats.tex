%!TEX root = ./main.tex

\section{Architectural Threats}
We now move to architectural threats.

\paragraph{Poorly Designed Consensus Algorithms.}  Blockchains could use consensus algorithms that are improperly vetted.  Perhaps performance degrades under adversarial circumstances, or even safety or liveness totally break.

\paragraph{Self-Invented Cryptography.}  Developers could attempt their own cryptographic algorithms.  While developers don't usually try to build things from scratch, they often try to ``optimize'' or ``extend'' things that break the scheme.  These things have happened multiple times in the HLF.

\paragraph{Academic Work is Broken.}  We could be affected by major design flaws in existing work that we implement.  

\paragraph{Poor Definitions.}  What do privacy and confidentiality on a blockchain mean anyway?  If they aren't defined rigorously, chances are any suggestions of privacy and confidentiality are meaningless.  Even if they are well-defined, the guarantees can sometimes be meaningless when looked at closely.  We need consistent, clear definitions of security.

\paragraph{Conformance for Blockchain.}  Another area that needs definition is integrity protection of generated blockchain for immutability.

\paragraph{Network Secrecy.}  Adversary can capitalize on insecure connection across components. Note that using tcp connections, merely sending a byte array, encoding the data before transmission are not security measures.

\paragraph{Network Availability.}  Adversary could intend to break the normal operations of the system through unwanted traffic sent to the network or blocking subset of communication flows.

\paragraph{Key Management.}  Private and/or Secret Keys play a critical role in protecting the sanctity of whole blockchain network. Improper key management, leaving these behind in a temporary storage, persisting them in plain text would open up attack surface for an adversary.

\paragraph{Expired Credentials.}  Adversary can utilize the permissions granted to an expired identity to gain access of the whole or part of the system.

\paragraph{Excessive Backwards Compatibility}  Engineers, developers being instructed to continue supporting known security holes as features due to the fix causing breaking changes in existing deployments or making new deployments less convenient to perform and/or harder to sell to begin with.
