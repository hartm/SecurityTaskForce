%!TEX root = ./main.tex

\section{Infrastructure Threats}

In this section, we begin by listing some infrastructural threats.  Note that \emph{most of these threats are common to many open source projects}.  This does not mean we should not care about or list them:  we need to be on the vanguard of security.  We need to be able to weight these threats appropriately and figure out how we should spend our security resources (cash and people's time). \textcolor{red}{Hart: we aren't even doing some of these correctly.}

\paragraph{Malicious Developer.}  A malicious adversary begins contributing to some code but eventually puts in backdoors for future system compromises.

\paragraph{Software Delivery MitM.}  An adversary figures out a way to convince third parties to accept malicious software packages as valid Hyperledger software.

\paragraph{Upstream Bugs.}  We could be affected by bugs in software that we use (e.g. heartbleed~\cite{durumeric2014matter} if we had consumed OpenSSL at the time). \textcolor{red}{Arun: are you intending to say the bugs in software packages one chooses in the infrastructure?}

\paragraph{Compromised Software Build Process.}  This could occur due to improper user access management or misconfigurations.

\paragraph{Leaked Credentials.} Developers unintentionally leave behind traces of secrets and commit them on a source code version control systems such as GitHub.

\paragraph{Improper Examples.} Developers who are new will refer to the open-source available sample codebase to implement and get their solution to production. The examples do not call out security risks.

\paragraph{Insufficient Testbed Setup.} Complex scenarios demand hardware resources that is not available at the moment, however an adversary with unlimited resource can be ahead in time.

\paragraph{Misusing Privileged Access.} Principle of least privilege should be followed, otherwise having extra access could lead to compromising a majority of the system.

\paragraph{Anonymous Access.} Sometimes the adversary is just interested in either denial of service or wants to misuse the resources available for anonymous users.

\paragraph{Chain Split or Halt}  This is an attack against the operational state of the network, where the side effects of an unhealthy system are valuable to the attacker.
A malicious attacker could use a bug or other aspect of the system to cause the network to split into multiple networks and stay split.
An alternate form would be to cause a network to stop, removing enough participants so the chain will not advance.
These attacks can be temporary or permanent.
A temporary split can still allow other systems depending on the network to produce incorrect behaviour if they are observing a state that is later abandoned.
In extreme cases the temporary split may continue for long enough that the split network will form two independent networks.

\paragraph {Physical Intrusion}  A breach of the physical location where the hardware is operated (usually a data center) leads to either compromised or unavailable cloud service(s) which leads to the entire chain being down due to a single point of failure design flaw.
